The Down Syndrome Mental State Exam is a cognitive assessment we administer at each visit. It is designed to evaluate an individual’s current mental functioning. 

While lower percentile scores are generally interpreted as indicating lower performance, these scores do not take into consideration level of intellectual functioning and other important clinical factors including medical history. We generally look for change over time to help indicate whether a change in cognitive status is happening. 

Below are your percentiles and a brief description of what each category means. 

\begin{enumerate}
    \item \textbf{Visual-Spatial} refers to the skill to be able to interpret what we see and where we are. 
    \item \textbf{Memory} refers to the ability to remember information, people, or experiences
    \item \textbf{Language} refers to the use of words in order to share information or communicate with others. 
    
\end{enumerate}

\subsection{Down Syndrome Mental Status Exam (DSMSE)}